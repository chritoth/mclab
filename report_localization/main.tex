\documentclass[conference]{IEEEtran}
\IEEEoverridecommandlockouts
% The preceding line is only needed to identify funding in the first footnote. If that is unneeded, please comment it out.
\usepackage{cite}
\usepackage{amsmath,amssymb,amsfonts}
\usepackage{graphicx}
\usepackage{textcomp}
\usepackage{xcolor}
\def\BibTeX{{\rm B\kern-.05em{\sc i\kern-.025em b}\kern-.08em
    T\kern-.1667em\lower.7ex\hbox{E}\kern-.125emX}}
\begin{document}

\title{A Simple Indoor Localization App\\
{\large Mobile Computing Lab - Summer Term 2018}
}

\author{\IEEEauthorblockN{Christian Toth}
\IEEEauthorblockA{
\textit{Graz University of Technology}\\
Graz, Austria \\
christian.toth@student.tugraz.at}
\and
\IEEEauthorblockN{Timur Cerimagic}
\IEEEauthorblockA{
\textit{Graz University of Technology}\\
Graz, Austria \\
t.cerimagic@student.tugraz.at}
}

\maketitle


\section{Task Description}
Goal of this project was to make an indoor localization app based on the TUGraz Inffeldgasse 16 first floor plan. Idea was to use previously developed activity tracking system to detect movement and therefore know if the user is walking or is in the idle state. Throughout the history various methods were used for indoor localization and some papers mention using and RF-based user location a tracing system with KNN algorithm, WLAN location determination system with Bayesian filter, Log normal shadowing and indoor localization with particle filter. As a requirement for this task we were challenged to use Bayesian filter or the particle filter. We were not particularly fond of measuring signal strength of many access points inside the building and because of that, we used particle filter approach. That approach didn’t come without its burden. There were few things which we had to take into the consideration. Those are the following:
We were split between image recognition pattern for the walls and hardcoded walls
Optimal number of particles to generate
Optimal sensor settings and proper use of available android sensors
Ways of recording the motion in the right moment 
Filtering and sampling
Step counting
Sensor orientation vs access point RSSI signal strength orientation
And few other implementation remarks. In the following chapter you can see how we tackled all of those tasks.

\section{Methods}
Considering that localization problem in the manner of this project is complex set of individual units that need to work perfectly together to have a decent result, we needed to find an adequate set of sensors and methods to accomplish that. First, we needed to adapt activity tracking system so it can additionally measure steps and also direction in which user is pointed to. Idea behind that is that we detect situation where user changed from idle state to a motion state by observing the variance in the signal. If there is a significant motion, signal variance would be noticed. Furthermore, magnitude and the frequency of the signal are measured so we can acknowledge a walking pattern. Thresholds for the frequency and the magnitude are 0.5 and 0.9 respectively and are defined in an experimental way. 

Initial idea was to use RSSI signal strength of the fixed access points around the building for orientation, but we decided to go with the onboard sensors for the sake of simplicity, even though we knew that building structure is going to be a problem for the magnetometer. In the end, two sensors were used; accelerometer and magnetometer. Accelerometer for the motion tracking (same signal processing as in the previous exercise with AA filter, downsampling , FFT and DFT analysis) and magnetometer was used for estimating angle relative to the North.  All sensor readings were happening in the background on the one thread, which was constantly refreshing the buffer with new sensor data, so it can be used in the motion estimation. Multithreading was fully utilized for this purpose, so we can have fresh data always ready for the usage in motion estimation. 

Another part of the system is Particle filter, which in the beginning uniformly spreads all the particles and with the help of a floor plan and motion estimation moves particles and resamples them as user walks around. In the final version we used 1000 particles for the floor, but we also tested with 10000 particles and precision was not improved, just the latency was increased.  MAYBE YOU CAN ADD FEW CRUCIAL POINTS ON PARTICLE FILTER. MOVING PATRICLES AND RESAMPLING.
Last part of the system is the UI part where particle positions are represented on the floor plan. As the user stops, positions and weights of the particles are calculated and main UI thread refreshes the image. Particle representation is done by meter to pixel ratio. Considering that we downloaded the official floor plan, we had the ruler which we used to calculate pixels per meter. We had somewhere around 81 pixels per square meter and that was one of the reasons why we choose to go with the 1000 particles. Screen was cleared every time before new particles were drawn.

\section{Evaluation}
Initial evaluation indicates that mentioned methods and tools were fully applicable in terms of defined problem. Results indicate particle convergence within a 15 -20 meter walk. (We started inside the lab and after the initial 15-meter-long aisle and few steps to the left we converged) Biggest problem that we encountered was the building construction. It most certainly affected our magnetometer due many metal frames and fences. Also, due to its shape (long straight aisle) it wasn’t too convenient for the particle filter, but results were surprisingly positive. Building in the paper had more convenient structure and still it took around 50 meters to converge. (20m along y axis and 30m along x axis). Considering that we decided to go with the hardcoded version of the walls, there were no problems regarding that part and no uncertainty for the particle generation model, but in case of more complex building it would be time consuming to hardcode all the walls and obstacles and therefore we see place for improvement in image pattern recognition. From the hardware side, 2013 Samsung Galaxy S4 mini was able to deliver the background processing and the main UI thread update without significant latency. Place for improvement is seen in number of sensor readings per time frame and number of particles computed in the given space.  Main conclusion is that particle filter is a powerful tool for the indoor localization, which if used properly, could deliver enviable results with average resources.



\begin{thebibliography}{00}
\bibitem{particlefilter}  M.S. Arulampalam ; S. Maskell ; N. Gordon ; T. Clapp, "A Tutorial on Particle Filters for Online Nonlinear/Non-Gaussian Bayesian Tracking", IEEE Trans. Signal Processing, 2002.


\bibitem{zee}  K.K. Chintalapudi ; V. Padmanabhan ; R. Sen ; K. Chintalapudi, "Zee: Zero-Effort Crowdsourcing for Indoor Localization", Mobicom, 2012.

\bibitem{dft} Oppenheim, Alan V. and Schafer, Ronald W. Discrete-Time Signal Processing. Prentice Hall Press, 2009.

\end{thebibliography}


\end{document}

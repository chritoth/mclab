\documentclass[conference]{IEEEtran}
\IEEEoverridecommandlockouts
% The preceding line is only needed to identify funding in the first footnote. If that is unneeded, please comment it out.
\usepackage{cite}
\usepackage{amsmath,amssymb,amsfonts}
\usepackage{graphicx}
\usepackage{textcomp}
\usepackage{xcolor}
\def\BibTeX{{\rm B\kern-.05em{\sc i\kern-.025em b}\kern-.08em
    T\kern-.1667em\lower.7ex\hbox{E}\kern-.125emX}}
\begin{document}

\title{A Simple Indoor Localization App\\
{\large Mobile Computing Lab - Summer Term 2018}
\thanks{Identify applicable funding agency here. If none, delete this.}
}

\author{\IEEEauthorblockN{Christian Toth}
\IEEEauthorblockA{
\textit{Graz University of Technology}\\
Graz, Austria \\
christian.toth@student.tugraz.at}
\and
\IEEEauthorblockN{Timur Cerimagic}
\IEEEauthorblockA{
\textit{Graz University of Technology}\\
Graz, Austria \\
t.cerimagic@student.tugraz.at}
}

\maketitle


\section{Task Description}
Goal of this project was to make an indoor localization app based on the TUGraz Inffeldgasse 16 first floor plan. Idea was to use previously developed activity tracking system to detect movement and therefore know if the user is walking or is in the idle state. Throughout the history various methods were used for indoor localization and some papers mention using and RF-based user location a tracing system with KNN algorithm, WLAN location determination system with Bayesian filter, Log normal shadowing and indoor localization with particle filter. As a requirement for this task we were challenged to use Bayesian filter or the particle filter. We were not particularly fond of measuring signal strength of many access points inside the building and because of that, we used particle filter approach. That approach didn’t come without its burden. There were few things which we had to take into the consideration. Those are the following:
We were split between image recognition pattern for the walls and hardcoded walls
Optimal number of particles to generate
Optimal sensor settings and proper use of available android sensors
Ways of recording the motion in the right moment 
Filtering and sampling
Step counting
Sensor orientation vs access point RSSI signal strength orientation
And few other implementation remarks. In the following chapter you can see how we tackled all of those tasks.

\section{Evaluation}
Initial evaluation indicates that mentioned methods and tools were fully applicable in terms of defined problem. Results indicate particle convergence within a 15 -20 meter walk. (We started inside the lab and after the initial 15-meter-long aisle and few steps to the left we converged) Biggest problem that we encountered was the building construction. It most certainly affected our magnetometer due many metal frames and fences. Also, due to its shape (long straight aisle) it wasn’t too convenient for the particle filter, but results were surprisingly positive. Building in the paper had more convenient structure and still it took around 50 meters to converge. (20m along y axis and 30m along x axis). Considering that we decided to go with the hardcoded version of the walls, there were no problems regarding that part and no uncertainty for the particle generation model, but in case of more complex building it would be time consuming to hardcode all the walls and obstacles and therefore we see place for improvement in image pattern recognition. From the hardware side, 2013 Samsung Galaxy S4 mini was able to deliver the background processing and the main UI thread update without significant latency. Place for improvement is seen in number of sensor readings per time frame and number of particles computed in the given space.  Main conclusion is that particle filter is a powerful tool for the indoor localization, which if used properly, could deliver enviable results with average resources.



\begin{thebibliography}{00}
\bibitem{b1} G. Eason, B. Noble, and I. N. Sneddon, ``On certain integrals of Lipschitz-Hankel type involving products of Bessel functions,'' Phil. Trans. Roy. Soc. London, vol. A247, pp. 529--551, April 1955.
\bibitem{b2} J. Clerk Maxwell, A Treatise on Electricity and Magnetism, 3rd ed., vol. 2. Oxford: Clarendon, 1892, pp.68--73.
\bibitem{b3} I. S. Jacobs and C. P. Bean, ``Fine particles, thin films and exchange anisotropy,'' in Magnetism, vol. III, G. T. Rado and H. Suhl, Eds. New York: Academic, 1963, pp. 271--350.
\bibitem{b4} K. Elissa, ``Title of paper if known,'' unpublished.
\bibitem{b5} R. Nicole, ``Title of paper with only first word capitalized,'' J. Name Stand. Abbrev., in press.
\bibitem{b6} Y. Yorozu, M. Hirano, K. Oka, and Y. Tagawa, ``Electron spectroscopy studies on magneto-optical media and plastic substrate interface,'' IEEE Transl. J. Magn. Japan, vol. 2, pp. 740--741, August 1987 [Digests 9th Annual Conf. Magnetics Japan, p. 301, 1982].
\bibitem{b7} M. Young, The Technical Writer's Handbook. Mill Valley, CA: University Science, 1989.
\end{thebibliography}


\end{document}
